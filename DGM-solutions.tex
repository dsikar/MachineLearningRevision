\documentclass{article}
\usepackage{amsmath}
\usepackage[T1]{fontenc}
\usepackage[utf8]{inputenc}
\usepackage{witharrows}
\begin{document}

\textbf{When I turn on the car:} \\
p(B): battery is charged (B=\{0,1\})\\
p(F): there is fuel in the tank (F=\{0,1\})\\
p(G): fuel gauge moves (G=\{0,1\})\\
\\
p(G = 1|B-=1,F=1) = 0.8 \\
p(G = 1|B-=1,F=0) = 0.2 \\
p(G = 1|B-=0,F=1) = 0.2 \\
p(G = 1|B-=0,F=0) = 0.1 \\
p(B=1) = 0.9 \\
p(F=1) = 0.9 \\
From this we can derive: \\
p(G = 0|B-=1,F=1) = 0.2 \\
p(G = 0|B-=1,F=0) = 0.8 \\
p(G = 0|B-=0,F=1) = 0.8 \\
p(G = 0|B-=0,F=0) = 0.9 \\
p(B=0) = 0.1 \\
p(F=0) = 0.1 \\
\textbf{If the gauge does not move, what is the probability that the fuel tank is empty?} \\
\begin{align*}
p(F=0|G=0) & =\frac{p(G=0|F=0)p(F=0)}{p(G=0)} \\
    p(G=0|F=0) & = \sum_{B\in \{0,1\}} p(G=0|B,F=0)p(B) \\
    For B=0: \\
    & = p(G=0|B=0,F=0)p(B=0) \\    
    & = 0.9 * 0.1 \\
    & = 0.09 \\
    For B=1: \\
    &= p(G=0|B=1,F=0)p(B=1) \\   
    & = 0.8 * 0.9 \\
    & = 0.72 \\
p(G=0|F=0) & = 0.09 + 0.72 = \textbf{0.81} \\  
p(F=0) & = \textbf{0.1} \\
p(G=0) & =  \sum_{B\in \{0,1\}}  \sum_{F\in \{0,1\}}  p(G=0|B,F)  p(B)  p(F) \\ 
For B=0, F=0: \\
     & = p(G=0|B=0,F=0)  p(B=0)  p(F=0) \\
    &= 0.9 * 0.1 * 0.1 \\
    & = 0.009 \\
For B=0, F=1: \\
     & = p(G=0|B=0,F=1)  p(B=0)  p(F=1) \\
    &= 0.8 * 0.1 * 0.9 \\
    & = 0.072 \\
For B=1, F=0: \\
     & = p(G=0|B=1,F=0)  p(B=1)  p(F=0) \\
    &= 0.8 * 0.9 * 0.1 \\
    & = 0.072 \\ 
For B=1, F=1: \\
     & = p(G=0|B=1,F=1)  p(B=1)  p(F=1) \\
    &= 0.2 * 0.9 * 0.9 \\
    & = 0.162 \\    
p(G=0) &= 0.009 + 0.072 + 0.072 + 0.162 = \textbf{0.315} \\
p(F=0|G=0) & =\frac{0.81*0.1}{0.315} \\
p(F=0|G=0) & = 0.257 \\
\end{align*}
\end{document}