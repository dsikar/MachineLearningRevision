\section{Correlation}

The correlation, or \textit{correlation coefficient}, or \textit{Pearson correlation coefficient}  between two random variable vector, is measure of their linear dependence, $A$ and $B$ defined as:
\begin{equation}
\rho(A,B) = \frac{COV(A,B)}{\sigma_A \sigma_B} \label{correlation}    
\end{equation}


where $\sigma_A$ and $\sigma_B$ and the standard deviations of A and B respectively. In Matlab, function $\rho(A,B)$ returns a matrix of correlation coefficients for each pairwise variable combination:
$$
R = 
\begin{pmatrix}
\rho(A,A) & \rho(A,B)\\
\rho(B,A) & \rho(B,B)
\end{pmatrix}
$$
Since A and B are always directly correlated to themselves, the correlation matrix can be written as:
$$
R = 
\begin{pmatrix}
1 & \rho(A,B)\\
\rho(B,A) & 1
\end{pmatrix}
$$
\subsection{Intuition}
Correlation can be described as a normalized covariance. By dividing the covariance by the product of means, the covariance is scaled, resulting in a value in the range $\{-1,1\}$, provided standard deviations are greater than zero.  
Random variable vectors can still be identified as positively or negatively correlated. For example, vectors [1 2 3] and [-1 -2 -3], as already observed with covariance, can be described as inversely related, or having a negative covariance. Looking at two examples in code:
\begin{verbatim}
>> corrcoef([1 2 3], [-1 -2 -3])

ans =

     1    -1
    -1     1
\end{verbatim}
We see both vectors have the absolute maximum negative correlation. whereas vectors [1 2 3] and [2 4 6]:
\begin{verbatim}
>> corrcoef([1 2 3], [2 4 6])

ans =

     1     1
     1     1
\end{verbatim}
Have the maximum positive correlation, that is to say, correlation tells the degree by which vectors are related.

\subsection{Discussion}
Both covariance and correlation are a concise way to compare sets of numbers. Both describe if sets, or vectors in the cases demonstrated, are negatively or positively related. Covariance can be thought of as a magnitude, while correlation can be thought of as a degree. Comparing vectors [1 500 1000] and [-1 -500 -1000], intuitively it can be say both have a negative correlation, which values in the first vector increase, corresponding values in the second vector decrease. Computing covariance and correlation in code:
\begin{verbatim}
>> cov([1 500 1000], [-1, -500, -1000])

ans =

   1.0e+05 *

    2.4950   -2.4950
   -2.4950    2.4950
>> corrcoef([1 500 1000], [-1, -500, -1000])

ans =

     1    -1
    -1     1  
\end{verbatim}
It can be noted that covariance is a large number, while the correlation indicates a best possible negative correlation. Comparing vectors [1 2 4] and [-1 -2 -4]:
\begin{verbatim}
>> cov([1 2 4], [-1 -2 -4])

ans =

    2.3333   -2.3333
   -2.3333    2.3333
>> corrcoef([1 2 4], [-1 -2 -4])

ans =

     1    -1
    -1     1   
\end{verbatim}
We can still determine a negative covariance and correlation. Noting that covariance between vectors [1 2 4] and [-1 -2 -4] (-2.3333) is several order of magnitudes smaller than the covariance between vectors [1 500 1000] and [-1, -500, -1000] (-2.4950 * 1.0e+05).