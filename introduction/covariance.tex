\section{Covariance}
Covariance between two random variable vectors (TODO DEFINE) $A$ and $B$ can be defined as:
\begin{equation}
COV(A,B)=\frac{1}{N-1}\sum_{i=1}^N(A_i-\mu_A)(B_i-\mu_B)   \label{covariance}
\end{equation}


Where $\mu_A$ is the mean of $A$, and $\mu_B$ is the mean of $B$. Covariance indicates how two random variable vectors are related.
\subsection{Intuition}
Given two vectors, A [1 2 3] and B [2 4 6], we see that both sequences are increasing so both random variable vectors are said to have positive correlation. Calculating the value manually we have:
\begin{verbatim}
MeanA = (1 + 2 + 3) / 3 = 2
MeanB = (2 + 4 + 6) / 3 = 4
COV(A,B) = 1/2*( (1-2*2-4) + (2-2*4-4) + (3-2)*(6-4)
COV(A,B) = 2
\end{verbatim}
In code:
\begin{verbatim}
>> cov([1 2 3], [2 4 6])

ans =

     1     2
     2     4
\end{verbatim}
Noting Matlab cov function will return a covariance matrix consisting of pairwise covariance calculations between random variable vectors.
$$
C = 
\begin{pmatrix}
cov(A,A) & cov(A,B)\\
cov(B,A) & cov(B,B)
\end{pmatrix}
$$
If we calculate the covariance between vectors A [1 2 3] and B [-1 -2 -3]:
\begin{verbatim}
>> cov([1 2 3], [-1 -2 -3])

ans =

     1    -1
    -1     1    
\end{verbatim}
we see the covariance is equal to -1. That is to say $A$ and $B$ have a negative covariance, or are inversely related.
Finally we look at A [1 1 1] and B [2 2 2]. Intuitively it can be seen the values do not seem to be related. In code:
\begin{verbatim}
>> cov([1 1 1], [2 2 2])

ans =

     0     0
     0     0
\end{verbatim}
we see A and B are neither positively nor negatively related.