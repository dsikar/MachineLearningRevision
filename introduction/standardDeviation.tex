\section{Standard Deviation}
Standard deviation is the square root of the variance, defined as:
\begin{equation}
S=\sqrt{\frac{1}{N-1}\sum_{i=1}^N|A_i-\mu|^2} \label{standardDeviation}    
\end{equation}


typically expressed as greek letter $\sigma$, lower case \textit{sigma}.

\subsection{Intuition}
The standard deviation of the values contained in a vector $A$, express how \textit{far} any given value is from the \textit{mean}, or $\mu$ as seen.  
Looking at the mean for our three examples:
\begin{verbatim}
>> mean([1 1; 1 2; 1 30]')

ans =

    1.0000    1.5000   15.5000
\end{verbatim}
we see the vectors, or in this case, array columns, which greater variance, also have a greater mean. Looking at the standard deviation for the same array columns:
\begin{verbatim}
>> std([1 1; 1 2; 1 30]')

ans =

         0    0.7071   20.5061
\end{verbatim}
we see that the standard deviation increases with the variance.

\subsection{units}
Standard deviation is expressed in the same unit as the mean.

\subsection{Discussion}
With respect to a set of numbers, vector $A$ in the examples shown, it follows that mean, variance and standard deviation are a concise form to describe a set of numbers, where the values in vector $A$ are centered (mean) and how spread out (standard deviation) the values are. Such that, given a vector with mean = 2 and standard deviation = 0.5, the expectation is that values in the array would be  centered around 2, most values ranging from 1.5 to 2.5. Given a vector with mean 100 and standard deviation equal to 50, the values in the array would be centered around 100, expected to mostly range from 50 to 150.