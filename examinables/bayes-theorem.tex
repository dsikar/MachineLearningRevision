\section{Bayes Theorem}
From the simmetry property $p(X,Y) = p(Y,X)$ we obtain the following relationship between conditional probabilities:
\begin{align*}
    p(X,Y) & = p(Y,X) \\
    p(Y|X)p(X) & = p(X|Y)p(Y) \\
\end{align*}
thus:
\begin{equation}
    p(Y|X) = \frac{p(X|Y)p(Y)}{p(X)} \label{bayes-theorem}
\end{equation}

which is known as \textit{Bayes' theorem}. Using the \textit{sum rule}, the denominator can be expressed in terms of quantities appearing in the numerator:
\begin{align*}
    p(X) & = \sum_Yp(X|Y)p(Y)  \\
    posterior & \propto likelihood\hspace{1mm}\mathrm{x}\hspace{1mm}prior
\end{align*}
We can view the denominator as being the normalization constant required to ensure that the sum of the conditional probability on the left-hand size of \ref{bayes-theorem} over all values of $Y$ equals one.
\subsection{Solved exercise}
\textit{A test for salmonella is made available to chicken farmers. The test will correctly show a positive result for salmonella 95\% of the time. However the test also shows positive results 15\% of the time in salmonella free chickens. 10\% of chickens have salmonella.} \\
\textbf{If a chicken tests positive, what is the probability that it has salmonella?} \\
First we define the notation for what is known and what is being asked:\\
\setlength{\parindent}{10ex} \par
Let A be the presence of salmonella \par
Let B be a positive test \par
P(A|B) The probability that a chicken has salmonella given a positive test \par
P(A) = 0.1 \% chickens with salmonella \par
P(B|A) = 0.95 \% chickens that test positive in the presence of salmonella \par
P(B|\midtilde A) = 0.15 \% chickens that test positive in the absence of salmonella \par
\noindent
\begin{align*}
    P(A|B) & = \frac{P(B|A)P(A)}{P(B)} \\
    P(B) & = P(B|A)P(A) + P(B|\midtilde A)P(\midtilde A)\\
    P(A|B) & = \frac{P(B|A)P(A)}{P(B|A)P(A) + P(B|\midtilde A)P(\midtilde A)} \\
    P(A|B) & = \frac{(0.95*0.1)}{(0.95*0.1)+(0.15*0.9)} \\
    P(A|B) & = 0.413 \\
    P(A|B) & = 41\% \\
\end{align*}
\subsection{Exercises}
1. Suppose we know nothing about coins except that each tossing event produces heads with some unknown probability p or tails with probability 1-p. Your model of a coin has one parameter, p. You observe 100 tosses and there are 53 heads. What is p? How about if you only tossed the coin once and got heads? Is it reasonable to give a single answer if we don’t have much data? 
\linebreak[2] 2. A drugs manufacturer claims that its roadside drug test will detect the presence of cannabis in the blood (i.e. show positive for a driver who has smoked cannabis in the last 72 hours) 90\% of the time. However, the manufacturer admits that 10\% of all cannabis-free drivers also will test positive. A national survey indicates that 20\% of all drivers have smoked cannabis during the last 72 hours. \textbf{TBC - add tutorial exercises without solutions} \\
