\section{K-Means}
We start by considering the problem of identifying groups, or clusters, of data points in a multidimensional space. Suppose we have a data set $\{\mathrm{x}_i,\cdots,\mathrm{x}_n\}$ consisting
of $N$ observations of a random $D$-dimensional Euclidean variable $\mathrm{x}_n$. Our goal is to partition the data set into some number $K$ of clusters, where we shall suppose for the moment that the value of $K$ is given. \\ 
We define $\mu_k$ as a \textit{prototype} associated with the k-th cluter. \\
We define a \textbf{binary indicator variable} $r_{nk}=\{0,1\}$ which describes which of the clusters a data point $x_n$ is assigned to. \\
We define an objective (error) function, which we seek to minimise):
\begin{equation}
    J = \sum_{n=1}^N \sum_{k=1}^K r_{nk} ||\mathrm{x}_n-\mu_k||^2 \label{objective-function}
\end{equation}
We then apply the iterative algorithm: \\
1. Choose initial values for $\mu_k$ \\
2. Minimise J wrt binary indicator variable $r_{nk}$ \\
3. Minimise J wrt $\mu_k$ \\
4. Repeat 2-3 until convergence, that is until $\mu_k$ no longer changes \\
Algorithm details: \\
We choose $\mu_k$ randomly, though in practice domain knowledge would be used for this choice \\
We update the binary indicator variable $r_{nk}$ using the function: \\
$$
r_{nk} = \begin{cases}
               1,\hspace{1mm} if \hspace{1mm} k = argmin_j||\mathrm{x}_n-\mu_j||^2 \\
               0\hspace{1mm}otherwise
            \end{cases}
$$
That is to say, given two clusters, with two prototypes (means), we compare the distance of a given point to both means, and assign it to the cluster represented by the smallest mean to the tested point. \\
We update the mean $\mu_k$:
$$
\mu_k = \frac{\sum_n r_{nk}\mathrm{x}_n}{\sum_n r_{nk}}
$$
Noting that we obtain the expression for updating the mean $\mu_k$ from \ref{objective-function}, by fixing $k$, taking the first derivative, setting it equal to zero then solving for $\mu_k$:
\begin{align*}
    J & = \sum_{n=1}^N r_{nk} (\mathrm{x}_n-\mu_k)^2 \\
    \frac{d}{dJ}\sum_{n=1}^N r_{nk} (\mathrm{x}_n-\mu_k)^2 & = 0 \\
    2 \sum_{n=1}^N r_{nk} (\mathrm{x}_n-\mu_k) & = 0 \\
    2\sum_{n=1}^N r_{nk}\mathrm{x}_n - 2\sum_{n=1}^N r_{nk}\mu_k & = 0 \\
    \sum_{n=1}^N r_{nk}\mu_k & = \sum_{n=1}^N r_{nk}\mathrm{x}_n \\
    \mu_k & = \frac{\sum_n r_{nk}\mathrm{x}_n}{\sum_n r_{nk}}    
\end{align*}
So $\mu_k$ is the mean of the $k$-th cluster, thus the name k-means. A practical worked example: \\
\\
You have the following set of 2-dimensional data points: \{1.9, 1.9\}, \{0.9, 1.1\}, \{1.8, 2.0\}, \{0.8, 1.0\}, \{1.1, 0.9\}, \{2.0, 1.9\}, \{1.0, 0.9\}, \{1.9, 1.8\}. \\
\\
Apply K-means with K=2 clusters using the following initial values for cluster prototypes: $\mu_1$= \{1.0, 1.0\} and $\mu_2$= \{2.0, 2.0\}. \\
\\
Which are the values of the final prototypes for each cluster? \\
\\
To which cluster does each data point belong to? \\
\\
Since step 1 is given, we go on to step 2: \\
To calculate we take the sum of the difference of squares between the data point and the mean coordinates: \\
For $x_1$ and $\mu_1$ we have $\Delta_1 = (1.9-1.0)^2+(1.9-1.0)^2 = 1.62$, and for $x_1$ and $\mu_2$ we have $\Delta_2 = (1.9-2.0)^2+(1.9-2.0)^2 = 0.02$, repeating for all:
\begin{verbatim}
u1= {1.0, 1.0} and u2= {2.0, 2.0}
x1 {1.9, 1.9}
(1.9-1.0)^2 + (1.9-1.0)^2 = 1.62
(1.9-2.0)^2 + (1.9-2.0)^2 = 0.02
x2 {0.9, 1.1}
(0.9-1.0)^2 + (1.1-1.0)^2 = 0.02
(0.9-2.0)^2 + (1.1-2.0)^2 = 2.02
x3 {1.8, 2.0}
(1.8-1.0)^2 + (2.0-1.0)^2 = 1.64
(1.8-2.0)^2 + (2.0-2.0)^2 = 0.04
x4 {0.8, 1.0}
(0.8-1.0)^2 + (1.0-1.0)^2 = 0.04
(0.8-2.0)^2 + (1.0-2.0)^2 = 2.44
x5 {1.1, 0.9}
(1.1-1.0)^2 + (0.9-1.0)^2 = 0.02
(1.1-2.0)^2 + (0.9-2.0)^2 = 2.02
x6 {2.0, 1.9}
(2.0-1.0)^2 + (1.9-1.0)^2 = 1.81
(2.0-2.0)^2 + (1.9-2.0)^2 = 0.01
x7 {1.0, 0.9}
(1.0-1.0)^2 + (0.9-1.0)^2 = 0.01
(1.0-2.0)^2 + (0.9-2.0)^2 = 2.21
x8 {1.9, 1.8}
(1.9-1.0)^2 + (1.8-1.0)^2 = 1.45
(1.9-2.0)^2 + (1.8-2.0)^2 = 0.05    
\end{verbatim}
so now we assign data points to binary indicator variables:
\begin{center}
 \begin{tabular}{||c c c c||} 
 \hline
 $\Delta_1$ & $\Delta_2$ & $r_1$ & $r_2$  \\ [0.5ex] 
 \hline\hline
 1.62 & 0.02 & 0 & 1 \\ 
 \hline
 0.02 & 2.02  & 1 & 0 \\ 
 \hline
1.64 & 0.04 & 0 & 1 \\
 \hline
0.04 & 2.44 & 1 & 0 \\
 \hline
 0.02 & 2.02 & 1 & 0 \\ 
  \hline
1.81 & 0.01 & 0 & 1 \\
 \hline
0.01 & 2.21 & 1 & 0 \\
 \hline
 1.45 & 0.05 & 0 & 1 \\ [1ex] 
 \hline
\end{tabular}
\end{center}
and recompute the means:
\begin{verbatim}
new u1 x2+x4+x5+x7/4 
{0.9, 1.1} + {0.8, 1.0} + {1.1, 0.9} + {1.0, 0.9} 
= {3.8, 3.9} / 4 = u1 = {0.95, 0.975}
new u2 x1+x3+x6+x8/4 
{1.9, 1.9} +  {1.8, 2.0} + {2.0, 1.9} + {1.9, 1.8} = 
{7.6, 7.6} / 4 = u2 = {1.9, 1.9}
\end{verbatim}
and repeat step 2 with the new means:
\begin{verbatim}
u1= {0.95, 0.975} and u2= {1.9, 1.9}.
x1 {1.9, 1.9}
(1.9-0.95)^2 + (1.9-0.975)^2 = 1.7581
(1.9-1.9)^2 + (1.9-1.9)^2 = 0
x2 {0.9, 1.1}
(0.9-0.95)^2 + (1.1-0.975)^2 = 0.0181
(0.9-1.9)^2 + (1.1-1.9)^2 = 1.64
x3 {1.8, 2.0}
(1.8-0.95)^2 + (2.0-0.975)^2 = 1.7731
(1.8-1.9)^2 + (2.0-1.9)^2 = 0.02
x4 {0.8, 1.0}
(0.8-0.95)^2 + (1.0-0.975)^2 = 0.0231
(0.8-1.9)^2 + (1.0-1.9)^2 = 2.02
x5 {1.1, 0.9}
(1.1-0.95)^2 + (0.9-0.975)^2 = 0.0281
(1.1-1.9)^2 + (0.9-1.9)^2 = 1.64
x6 {2.0, 1.9}
(2.0-0.95)^2 + (1.9-0.975)^2 = 1.9581
(2.0-1.9)^2 + (1.9-1.9)^2 = 0.01
x7 {1.0, 0.9}
(1.0-0.95)^2 + (0.9-0.975)^2 = 0.0081
(1.0-1.9)^2 + (0.9-1.9)^2 = 1.81
x8 {1.9, 1.8}
(1.9-0.95)^2 + (1.8-0.975)^2 = 1.5831
(1.9-1.9)^2 + (1.8-1.9)^2 = 0.01
\end{verbatim}
then recompute $r_1$ and $r_2$:
so now we assign data points to binary indicator variables:
\begin{center}
 \begin{tabular}{||c c c c||} 
 \hline
 $\Delta_1$ & $\Delta_2$ & $r_1$ & $r_2$  \\ [0.5ex] 
 \hline\hline
 1.7581 & 0 & 0 & 1 \\ 
 \hline
 0.0181 & 1.64  & 1 & 0 \\ 
 \hline
1.7731 & 0.02 & 0 & 1 \\
 \hline
0.0231 & 2.02 & 1 & 0 \\
 \hline
 0.0281 & 1.64 & 1 & 0 \\ 
  \hline
1.9581 & 0.01 & 0 & 1 \\
 \hline
0.0081 & 1.81 & 1 & 0 \\
 \hline
 1.5831 & 0.01 & 0 & 1 \\ [1ex] 
 \hline
\end{tabular}
\end{center}
As the output is the same (same binary indicator variables obtained), the means will not change and convergence has been achieved, giving the solution: \\
$\mu_i$ = \{0.95,0.975\}, $\mu_2$ = \{1.9, 1.9\} \\
$r_1$ = \{0,1,0,1,1,0,1,0\} \\
$r_2$ = \{1,0,1,0,0,1,0,1\} \\