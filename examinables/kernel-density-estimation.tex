\section{Kernel Density Estimation and K-Nearest Neighbours}
From histograms, we have seen that data can be split into bins of width $\Delta$, that that the probability value for a bin is given by:
$$
    p_i=\frac{n_i}{N\Delta_i}
$$
If we consider the parameter $\Delta$ to represent a volume, and write the parameter $n_i$ as $K$, representing the number of points falling inside a region, and the parameter $p_i$ as $p(x)$, representing a density we are trying to approximate, we can rewrite the equation as
\begin{equation}
    p(x)=\frac{K}{NV} \label{density-estimation}
\end{equation}
If we fix $V$ and determine K, that is the Kernel Density Estimation approach. If we fix $K$ and estimate $V$, that is the K-Nearest-Neighbours approach.\\
If we consider a region $R$ and wish to count the number K of points falling within this region, we use a \textit{Parzen window}:
$$
  k((\mathrm{x}-\mathrm{x}_n)/h) = \begin{cases}
               1,\hspace{1mm}|(x_i-x_{ni})/h|\hspace{1mm} \leq\hspace{1mm} 1/2, \hspace{1mm}i = 1,\cdots, D, \\
               0\hspace{1mm}otherwise
            \end{cases}
$$
where $\mathrm{x}$ is the hypercube centre, and $h$ is the height. To count the number of points in a hypercube:
\begin{equation}
K = \sum_{n=1}^Nk\bigg(\frac{\mathrm{x}-\mathrm{x}_n}{h}\bigg)
\end{equation}
Substituting in \ref{density-estimation} and changing $V$ for $h^D$ to represent the volume of a hypercube of side $h$ in $D$ dimensions, we have \textbf{kernel density estimator}:
\begin{equation}
    p(x) = \frac{1}{N}\sum_{n=1}^N \frac{1}{h^D} k\bigg(\frac{\mathrm{x}-\mathrm{x}_n}{h}\bigg)
\end{equation}
We now look at the \textbf{K nearest neighbour} method for density estimation. We return to \ref{density-estimation}, fixing the value of $K$ then finding an appropriate value of $V$. We consider a sphere centred on the point $\mathrm{x}$ at which we wish to estimate the density $p(x)$, and we allow the radius of the sphere to grow until it contains precisely $K$ data points. the estimate of the density is then given by \ref{density-estimation} with $V$ set to the volume of the resulting sphere.\\
The technique can now be extended to address the problem of classification. To do this, we
apply the \textit{K-nearest-neighbour} density estimation technique to each class separately
and then make use of Bayes’ theorem. Let us suppose that we have a data set comprising
$N_k$ points in class $C_k$ with $N$ points in total, so that $\sum_kN_k = N$. If we
wish to classify a new point $x$, we draw a sphere centred on $x$ containing precisely
$K$ points irrespective of their class. Suppose this sphere has volume $V$ and contains
$K_k$ points from class $C_k$. Then \ref{density-estimation} provides an estimate of the density associated with each class
$$
p(\mathrm{x}|C_k) = \frac{K_k}{N_kV}
$$
remembering that $K_k$ are the total number of points belonging to class $C_k$ in region $V$, and $N_k$ is the total number of points belonging to class $C_k$. The unconditional density is given by 
$$
p(\mathrm{x}) = \frac{K}{NV}
$$
while the class priors are given by
$$
p(C_k) = \frac{Nk}{N}.
$$
We can combine the three previous equations and use Bayes' theorem  to obtain the posterior probability of class membership
$$
p(C_k|\mathrm{x}) = \frac{(p(\mathrm{x}|C_k)p(C_k)}{p(\mathrm{x})}=\frac{K_k}{K}
$$


